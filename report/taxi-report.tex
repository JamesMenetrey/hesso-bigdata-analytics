\documentclass[a4paper]{article}
\usepackage[T1]{fontenc}
\usepackage[utf8]{inputenc}
\usepackage[margin=1in]{geometry}
\usepackage[english]{babel}
\usepackage{charter}
\usepackage{graphicx}
\usepackage{parskip}
\usepackage{mathtools}
\usepackage{amssymb}
\usepackage{xfrac}
\usepackage{booktabs}
\usepackage{longtable}
\usepackage{minted}


\usepackage{setspace} % increase interline spacing slightly
\setstretch{1}

\def\arraystretch{1} % increase tabular cells padding

\title{
\Large{Big Data Analytics 2019}\\
\huge{NY City Taxi Data Analysis}\\
\large{Final laboratory}}

\author{
Prof. Marcel Graf\\
Prof. Nastaran Fatemi\\
\\
Romain Claret, Jämes Ménétrey and Damien Rochat\\
Master of Science in Engineering\\
HES-SO University of Applied Sciences and Arts
\date{\today}}

\begin{document}
\maketitle

\tableofcontents
\clearpage

\section{Introduction}
The goal of this laboratory is to identify the most used areas from New York City, from a taxi agency point of view, to open new agency locations, in order to increase profit and customer satisfaction. This analysis aims to highlight two topics of interest: (1) the best locations to find out customers, and (2) identify the most profitable spots in New York City, which are the longest trips.


\section{Dataset}
The dataset, provided by Chris Whong, is the list of trips and fares of the taxi in New York City. It is composed of 2 parts, which are themselves fragmented into 13 CVS. These parts are the trips related information (trip\_data.7z, sized 1.72GB) and the fares related to each trip (trip\_fare.7z, sized 4.1GB). Each trip contains the following information:

\begin{itemize}
  \item Taxi identification
  \item Driver identification
  \item Start time and end time
  \item GPS coordinates of pick up and drop off
  \item Fare information (amount, payment type, tip)
\end{itemize}


\section{Pre-processing and features selection}


\section{Machine learning}
\subsection{Algorithms}
\subsection{Optimizations}

\section{Testing and evaluation}

\section{Results}

\section{Conclusion}
\subsection{Next steps}

\end{document}
























